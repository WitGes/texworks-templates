%&program=xetex
%&encoding=UTF-8 Unicode

\baselineskip=14pt
\tolerance=2000
\font\tenrm="Minion Pro" at 10pt \tenrm
\font\bigbold="Minion Pro/B" at 24pt
\def\TeX{T\lower.5ex\hbox{\kern-.1667em E}\kern-.125em X}
\def\XeTeX{X\kern-.1667em\lower.5ex\hbox{Ǝ}\kern-.15em \TeX}
\centerline{\bigbold Pictures in \XeTeX}
\smallskip
\centerline{\XeTeXpicfile "xetex.jpg"}
\smallskip
\def\\{\char`\\}

This file demonstrates simple use of the \\XeTeXpicfile command
in \XeTeX\ to include pictures in a typeset document.  When \XeTeX\
reads a picture file using \\XeTeXpicfile, it knows the dimensions of
the picture, which means it is possible to use macros to automatically
create a suitable “hole” in the text for the picture to go in.

\def\leftpicpar#1{\setbox0=\hbox{\XeTeXpicfile #1}%
  \dimen0=\wd0 \advance\dimen0 by 3pt
  \count255=\ht0 \advance\count255 by \baselineskip
    \divide\count255 by \baselineskip
  \hangindent\dimen0 \hangafter-\count255
  \noindent\llap{\vbox to 0pt{\kern-0.7\baselineskip\box0\vss}\kern3pt}%
  \indent\ignorespaces}

\leftpicpar{"xetex.jpg" scaled 400}
This shows one possible approach (rather crudely implemented) for
formatting text to fit around a picture, making use of \XeTeX’s
ability to measure the picture file.  In practice, it would be much
better to implement some higher-level picture placement macros, with
simple parameters to control various aspects of positioning and text
runaround, but this is left as an exercise for the reader.  Now we
just need enough text to fill several lines, so we can check that it
wraps around the picture properly.

Note that in versions of \XeTeX\ prior to 0.5, there was a primitive
simply called \\picfile, which was always enabled. Beginning from
release 0.5, this command has been renamed \\XeTeXpicfile, and is only
available when running in “extended” mode (similarly to e-\TeX\
extensions to the original \TeX\ language). Also, the \\picfile\
command recognized keywords “xshifted” and “yshifted” for moving the
image within its box. These keywords are no longer supported, as the
same effect can be achieved with \TeX\ macros.

There is a similar command \\XeTeXpdffile\ for including pages from
existing PDF documents as graphics within a \XeTeX\ document.

\bye


To support graphics inclusion, XETEX provides the \XeTeXpicfile command.
Normally, in other TEX systems, this is done with the \special command. How-
ever, \XeTeXpicfile offers the advantage that XETEX can determine the size of
the picture, and thus macros can be written to fit the picture to a certain space or
create the needed space for the picture, bypassing the manual picture-measuring
process otherwise needed. Specifically, \XeTeXpicfile is used as follows:

       \XeTeXpicfile <filename> [options]

where <filename> is the name (full pathname, or relative to the main document) of
a graphics file. Most graphic file formats are supported (using QuickTime's Graphic
Importer facility).

The [options] in the \XeTeXpicfile command use the following keywords:

       width <dimen>
       height <dimen>
       scaled <scalefactor>
       xscaled <scalefactor>
       yscaled <scalefactor>
       rotated <degrees>

Scaling and rotation operations are executed in the order specified. If one of width
or height, but not the other, is specified at any stage, the other dimension is scaled
in proportion. If both absolute size and scaled are used at the same stage in the
transformation sequence, not separated by a rotation, the size is set as requested
and the scale factor is ignored.

The final result of a \XeTeXpicfile command is an object which can be
included in the text being typeset just like a character, albeit usually a rather large
one. It could be thought of as a box, though it cannot be unboxed. It can, however,
be put inside a normal TEX box register, and in fact this is often desirable. After
saying, for example,

        \box0=\hbox{\XeTeXpicfile "my-pic.jpg" scaled 750 rotated 45}

you can find the dimensions of the graphic by examining the width and height of
box #. (The depth will always be zero.) Your macros can then arrange to place the
picture suitably on the page. (See the PicFileSample.tex file.)

There is also a \XeTeXpdffile command. This is similar to \XeTeXpicfile,
but recognizes an additional option:

        page <number>

This specifies the page to be included from the PDF file (default is the first page).
Negative numbers can be used to indicate pages relative to the end of the file;
thus page -2 means the second to last page. The page option, if used, must be
specified as the first keyword after the filename, before any transformations.

Note that PDF files can also be included by the \XeTeXpicfile command,
but this has two significant drawbacks: they will be rendered as lower-resolution
graphics, and only the first page is available.


% !TEX TS-program = xelatex
% !TEX encoding = UTF-8

\documentclass[a4paper,12pt]{article}

\usepackage{fontspec,xunicode}
\defaultfontfeatures{Ligatures=TeX,Scale=MatchLowercase}
\usepackage{mathpazo}

\setmainfont{TeX Gyre Bonum}
%\setsansfont{TeX Gyre Heros}
%\setmonofont{DejaVu Sans Mono}

\begin{document}
\pagestyle{empty}

\section*{The basics of the \textsf{fontspec} package}

The \textsf{fontspec} package enables automatic font selection
for \LaTeX{} documents typeset with Xe\TeX{} or Lua\TeX.
The basic command is
\begin{center}
\verb|\fontspec[font features]{font display name}|
\end{center}
The default, sans serif, and typewriter fonts may be set with the
\verb|\setmainfont|, \verb|\setsansfont| and \verb|\setmonofont|
commands, respectively, as shown in the preamble. They take the
same syntax as the \verb|\fontspec| package. All expected font
shapes are available:
\begin{center}
  {\itshape Italics and \scshape small caps\dots}\\
  {\sffamily\bfseries Bold sans serif and \itshape bold italic sans serif\dots}
\end{center}
Text fonts in maths mode are also changed (e.g., notice the cosine function in
$$
  \cos(n\pi)=\pm 1 \hbox{ dla $n=1$, $2$, \ldots}
$$
but only if the roman and sans serif fonts are set in
the preamble; \verb|\setmainfont| will not affect these maths mode fonts when
called mid-document.
Maths symbols themselves are not affected
$$
  \int_1^\infty {1\over x^2}\,dx = 1
$$

Please see the complete \emph{fontspec} documentation for further information.

\end{document}

% !TEX TS-program = xelatex
% !TEX encoding = UTF-8

\documentclass[a4paper,12pt]{article}

\usepackage{fontspec,xunicode}
\defaultfontfeatures{Ligatures=TeX,Scale=MatchLowercase}

\setmainfont{TeX Gyre Bonum}
%\setmainfont{TeX Gyre Termes}
%\setmmonofont{DejaVu Sans Mono}

\usepackage{graphicx} % ftp://ftp.tex.ac.uk/tex-archive/info/epslatex.pdf

\usepackage{polyglossia}
\setdefaultlanguage{polish}

\usepackage[parfill]{parskip} % akapity oddzielamy pustym wierszem

\renewcommand{\labelitemi}{--}

\title{Otoczenia \LaTeX-owe}
\date{28.02.2012}

\begin{document}
\pagestyle{empty}
%\maketitle
%\tableofcontents
%\newpage

\section*{Otoczenia itemize i~enumerate}

Jakaś wyliczanka:

\begin{itemize}
\item Ra, dwa, trzy
\item Baba Jaga patrzy!
\end{itemize}


\section*{Otoczenie figure}

Więcej tekstu\ldots, jeszcze więcej tekstu\ldots


\section*{Otoczenie table, tabular}

Więcej tekstu\ldots, jeszcze więcej tekstu\ldots

\section*{Otoczenie verbatim}

Krótkie wstawki ,,verbatim'': \verb|!@#$%^&*|.


\end{document}

Otoczenie center
Deklaracja \centering
Polecenie \includegraphics
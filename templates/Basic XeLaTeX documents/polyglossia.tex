% !TEX TS-program = xelatex
% !TEX encoding = UTF-8

% Dokumentacja:
%
%  polyglossia  -- http://ctan.org/tex-archive/macros/xetex/latex/polyglossia
%  fontspec     -- http://ctan.org/tex-archive/macros/xetex/latex/fontspec

\documentclass[a4paper,12pt]{article}

\usepackage{fontspec}
\defaultfontfeatures{Ligatures=TeX,Scale=MatchLowercase}
\setmainfont{DejaVu Serif}

\usepackage{polyglossia}
\setdefaultlanguage{polish}

\title{Oskary}
\author{L. Aureat}
\date{27.02.2012}

\begin{document}
\maketitle
\tableofcontents

\newpage

\section{Oskarowa gala}

,,Radość, wzruszenie, zaskoczenie --- to też plejada gwiazd tego
wieczoru. Skąd w tym towarzystwie łzy? Przecież to niezwykle radosna
uroczystość.'' (\emph{Polityka}, 27.02.2012)

\subsection{Dlaczego zdobywcy Oskarów płaczą?}

,,Nie dziwmy się zatem, że ktoś płacze, choć płacz wydaje nam się
reakcją nieadekwatną. Ma swoje powody --- po prostu uwalnia napięcie.''
(\emph{Polityka}, 27.02.2012)

\newpage

\section{Oskarowa gala}

,,Radość, wzruszenie, zaskoczenie --- to też plejada gwiazd tego
wieczoru. Skąd w tym towarzystwie łzy? Przecież to niezwykle radosna
uroczystość.'' (\emph{Polityka}, 27.02.2012)

\subsection{Dlaczego zdobywcy Oskarów płaczą?}

,,Nie dziwmy się zatem, że ktoś płacze, choć płacz wydaje nam się
reakcją nieadekwatną. Ma swoje powody --- po prostu uwalnia napięcie.''
(\emph{Polityka}, 27.02.2012)

\end{document}

Środowiska:

  figure
  table
  tabular

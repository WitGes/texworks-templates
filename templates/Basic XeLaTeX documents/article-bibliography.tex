% !TEX TS-program = xelatex
% !TEX encoding = UTF-8 Unicode

\documentclass[a4paper,12pt]{article}

\usepackage{fontspec,xunicode}
\defaultfontfeatures{Ligatures=TeX,Scale=MatchLowercase}
\setmainfont{TeX Gyre Bonum}

% polecenie \includegraphics, dokumentacja „epslatex.pdf”
\usepackage{graphicx}

\usepackage{polyglossia}
\setdefaultlanguage{polish}

% przełamywanie url
\usepackage{url}

\bibliographystyle{unsrt}

\title{Bibliografia}
\author{Ja}
\date{28 lutego 2012}

\begin{document}
\maketitle
\tableofcontents

\section{Podstawy}\label{sec:basics}

W fontach maszynowych, ang. \emph{monospaced fonts},
wszystkie znaki mają taką samą  szerokość. Za pomocą takich
fontów składane są listingi lub kod programów.

\section{Poziom średnio zaawansowany}\label{sec:intermediate}

W książce ,,\LaTeX\ Web Companion'' opisano
format \LaTeX\ i podstawowe pakiety do niego.

\section{Rzeczy zaawansowane}\label{sec:advanced}

DocBookXsl jest ciekawym projektem. Więcej na ten temat
można poczytać w~\cite{wiki.docbookxsl}.

Do konwersji dokumentu z formatu latex na html najlepszym narzędziem
jest TeX4ht~\cite[uwaga, system szuka nowego opiekuna]{Gurari.TeX4ht}.

O fontach maszynowych już było na stronie~\pageref{sec:basics}.

\bibliography{ppn}

\end{document}


Kompilacja:

  xelatex article-bibliography
  bibtex
  xelatex article-bibliography
  xelatex article-bibliography
